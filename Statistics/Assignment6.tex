\documentclass{article}
\usepackage{setspace}
\usepackage{amsmath}
\usepackage{amssymb}
\usepackage{amsthm}
\usepackage{amsfonts}
\usepackage{graphicx} 
\usepackage{float} 
\usepackage{fancyhdr}                                
\usepackage{lastpage}        
\usepackage{textcomp}                               
\usepackage{layout}   
\usepackage{subfigure} 
\usepackage{geometry}
\geometry{a4paper,scale=0.8}
\pagestyle{fancy}  
\lhead{ZHANG HUAKANG}
\chead{Assignment 6} 
\rhead{DB92760} 
\renewcommand{\baselinestretch}{1.05}
\title{Assignment 6 of CISC 1006}
\author{ZHANG HUAKANG \\ DB92760 \\ \\ Computer Science, \\Faculty of Science and Technology}
\begin{document}
    \maketitle
    \section{}
        \paragraph{
            It easy to know that $x~Poission$ where $\lambda=3$.
            $$P(X=x)=\frac{\lambda^x}{x!}e^{-\lambda}$$
        }
        \subsection{}
            \paragraph{
                \begin{equation*}
                    \begin{split}
                        P(X=5)=&\frac{3^5}{5!}e^{-3}\\
                            \approx&0.1008\\
                    \end{split}
                \end{equation*}
            }
        \subsection{}
            \paragraph{
                \begin{equation*}
                    \begin{split}
                        P(X<3)=&\sum_{x=0}^2 P(X=x)\\
                            =&\sum_{x=0}^2 \frac{\lambda^x}{x!}e^{-\lambda}\\
                            \approx&0.4232
                    \end{split}
                \end{equation*}
            }
        \subsection{}
            \paragraph{
                \begin{equation*}
                    \begin{split}
                        P(X\geq 2)&=1-P(X\leq 1)\\
                            =&1-\sum_{x=0}^1 \frac{3^x}{x!}e^{-3}\\
                            \approx&0.8008\\
                    \end{split}
                \end{equation*}
            }
    \section{}
        \paragraph{
            It easy to know that $x~Poission$ where $\lambda=5$.
            $$P(X=x)=\frac{\lambda^x}{x!}e^{-\lambda}$$
        }
        \subsection{}
            \paragraph{
                \begin{equation*}
                    \begin{split}
                        P(X>5)=&1-P(X\leq 5)\\
                            =&1-\sum_{x=0}^5\frac{5^x}{x!}e^{-5}\\
                            \approx&0.3840\\
                    \end{split}
                \end{equation*}
            }
        \subsection{}
            \paragraph{
                \begin{equation*}
                    \begin{split}
                        P(\text{3 of next 4 days})=&C_4^3 P(X>5)^3 P(X\leq 5)\\
                            \approx&0.0349\\
                    \end{split}
                \end{equation*}
            }
        \subsection{}
        \paragraph{
            \begin{equation*}
                \begin{split}
                    P(\text{The first time in April on April 5th})=&P(X>5) P(X\leq 5)^4\\
                        \approx&0.0553\\
                \end{split}
            \end{equation*}
        }
    \section{}
        \subsection{}
            \paragraph{
                Using Binomial distribution:
                \begin{equation*}
                    \begin{split}
                        P(X<5|\text{In 2000 people})=&\sum_{x=0}^4C_{2000}^x 0.002^x(1-0.002)^{2000-x}\\
                            \approx&0.6288\\
                    \end{split}
                \end{equation*}
                Using Poission Approximation:
                \begin{equation*}
                    \begin{split}
                        \lambda=&np\\
                            =&2000\times 0.002\\
                            =&4\\
                        P(X=x)=&\frac{4^x}{x!}e^{-4}\\
                        P(X<5)=&\sum_{x=0}^4\frac{4^x}{x!}e^{-4}\\
                            \approx&0.6288\\
                    \end{split}
                \end{equation*}
            }
        \subsection{}
            \paragraph{
                Using Binomial distribution:
                \begin{equation*}
                    \begin{split}
                        P(X=x)=&C_{2000}^x 0.002^x(1-0.002)^{2000-x}\\
                    \end{split}
                \end{equation*}
                Using Poission Approximation:
                \begin{equation*}
                    \begin{split}
                        P(X=x)=&\frac{4^x}{x!}e^{-4}\\
                    \end{split}
                \end{equation*}
            }
            \subsubsection{}
                \paragraph{
                    Using Binomial distribution:
                    \begin{equation*}
                        \begin{split}
                            \mu=&\sum_{x=0}^{2000}xP(X=x)\\
                                =&\sum_{x=0}^{2000}xC_{2000}^x0.002^x(1-0.002)^{2000-x}\\
                                =&3.99\dot{9}\\
                                \approx&4.0000\\
                        \end{split}
                    \end{equation*}
                    Using Poission Approximation:
                    \begin{equation*}
                        \begin{split}
                            &\text{By definition:}\\
                            &\mu=4.000\\
                        \end{split}
                    \end{equation*}
                }
            \subsubsection{}
                \paragraph{
                    Chebyshev's inequality:
                    $$
                    P(|X-\mu|\geq k\sigma)\leq \frac{1}{k^2}
                    $$
                    and
                    $$
                    P(|X-\mu|\leq k\sigma)\geq 1-\frac{1}{k^2}.
                    $$
                    \begin{equation}
                        \begin{split}
                            X\geq 1500\\
                            |X-\mu|>1496\\
                            \sigma^2=&\lambda\\
                                =&4\\
                            \sigma=&2\\
                            1496=&k\sigma\\
                            k=&748\\
                            &\text{Thus}\\
                            P(|X-\mu|\geq 748\sigma)\leq& \frac{1}{748^2}\\
                                =&\frac{1}{559504}\\
                                \approx&1.787\times 10^{-6}\\
                        \end{split}
                    \end{equation}
                }
\end{document}
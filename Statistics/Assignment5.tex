\documentclass{article}
\usepackage{setspace}
\usepackage{amsmath}
\usepackage{amssymb}
\usepackage{amsthm}
\usepackage{amsfonts}
\usepackage{graphicx} 
\usepackage{float} 
\usepackage{fancyhdr}                                
\usepackage{lastpage}        
\usepackage{textcomp}                               
\usepackage{layout}   
\usepackage{subfigure} 
\usepackage{geometry}
\geometry{a4paper,scale=0.8}
\pagestyle{fancy}  
\lhead{ZHANG HUAKANG}
\chead{Assignment 5} 
\rhead{DB92760} 
\renewcommand{\baselinestretch}{1.05}
\title{Assignment 5 of CISC 1006}
\author{ZHANG HUAKANG \\ DB92760 \\ \\ Computer Science, \\Faculty of Science and Technology}
\begin{document}
    \maketitle
    \section{}
        \subsection{}
            We can konw the probability that $x$ trucks are fail
            $$P(X=x)=f(x)=C_15^x 0.25^x 0.75^{15-x}$$
            \subsubsection{}
                \begin{equation*}
                    \begin{split}
                        P(3\leq X\leq 6)=&\sum_{x=3}^6 C_15^x 0.25^x 0.75^{15-x}\\ 
                            \approx&0.225199065+0.225199065+0.165145981+0.091747767\\
                            \approx&0.7073\\                            
                    \end{split}
                \end{equation*}
            \subsubsection{}
                \begin{equation*}
                    \begin{split}
                        P(X<4)=&\sum_{x=0}^3 C_15^x 0.25^x 0.75^{15-x}\\
                            \approx&0.4613\\
                    \end{split}
                \end{equation*}
            \subsubsection{}
                \begin{equation*}
                    \begin{split}
                        P(X>5)=&1-P(X\leq 5)\\
                            \approx&1-0.851631923\\
                            \approx&0.1484\\
                    \end{split}
                \end{equation*}
        \subsection{}
            \subsubsection{}
                \begin{equation*}
                    \begin{split}
                        \mathbb{E}[X]=&\sum_{x=0}^15 xf(x)\\
                            =&0+0.066817305+
                            0.31181409+
                            0.675597196+\\
                            &0.900796261+
                            0.825729906+
                            0.550486604+\\
                            &0.275243302+
                            0.104854591+
                            0.030582589+\\
                            &0.006796131+
                            0.001132688+
                            0.000137296+\\
                            &1.14413E-05+
                            5.86733E-07+
                            1.39698E-08\\
                            =&3.75\\
                    \end{split}
                \end{equation*}
            \subsubsection{}
                \begin{equation*}
                    \begin{split}
                        \mathbb{E}[X^2]=&\sum_{x=0}^15 x^2f(x)\\
                            =&0+
                            0.066817305+
                            0.62362818+\\&
                            2.026791587+
                            3.603185043+
                            4.128649528+\\&
                            3.302919623+
                            1.926703113+\\&
                            0.83883673+
                            0.275243302+
                            0.067961309+\\&
                            0.012459573+
                            0.001647547+
                            0.000148737+\\&
                            8.21427E-06+
                            2.09548E-07\\
                            =&16.875  \\
                        var(X)=&\mathbb{E}[X^2]-\mathbb{E}[X]^2\\
                            =&2.8125              
                    \end{split}
                \end{equation*}
    \section{}
        \subsection{}
            \subsubsection{}
                \begin{equation*}
                    \begin{split}
                        P_a=&\frac{C_{17}^3C_3^0}{C_{20}^3}\\
                            =&\frac{34}{57}\\
                            \approx&0.5965\\
                    \end{split}
                \end{equation*}
            \subsubsection{}
                \begin{equation*}
                    \begin{split}
                        P_b=&\frac{C_{19}^2C_1^1}{C_{20}^3}\\
                            =&\frac{17}{20}\\
                            =&0.8500
                    \end{split}
                \end{equation*}
        \subsection{}
            \subsubsection{}
                \begin{equation*}
                    \begin{split}
                        P_a=&C_3^3\frac{17}{20}^3\\
                            =&\frac{4913}{8000}\\
                            \approx&0.6141\\
                    \end{split}
                \end{equation*}
            \subsubsection{}
                \begin{equation*}
                    \begin{split}
                        P_b=&C_3^1\frac{1}{20}\\
                            =&\frac{3}{20}\\
                            =&0.1500
                    \end{split}
                \end{equation*}
    \section{}
        \subsection*{Hypergeometric}
            \begin{equation*}
                \begin{split}
                    P_H(X)=&\frac{C_{4000}^xC_{6000}^{15-x}}{C_{10000}^15}
                \end{split}
            \end{equation*}
        \subsection*{Binomial Approximation}
            We can use Binomial to approximate Hypergeometric, where $\theta=\frac{4000}{10000}=0.4$
            \begin{equation*}
                \begin{split}
                    P(X\leq 7)=&\sum_{x=0}^7 C_{15}^x \theta^x(1-\theta)^{15-x}\\
                        \approx&0.7869\\
                \end{split}
            \end{equation*}
    \section{}
        \subsection{}
            \subsubsection{}
                \begin{equation*}
                    \begin{split}
                        P_i=&C_3^00.8^0\times 0.2^3\\
                            =&\frac{1}{125}\\
                            =&0.0080
                    \end{split}
                \end{equation*}
            \subsubsection{}
                \begin{equation*}
                    \begin{split}
                        P_{ii}=&C_3^10.8^1\times 0.2^2\\
                            =&\frac{4}{125}\\
                            =&0.0960
                    \end{split}
                \end{equation*}
            \subsubsection{}
                \begin{equation*}
                    \begin{split}
                        P_{iii}=&C_3^20.8^2\times 0.2^1+C_3^30.8^3\times 0.2^0\\
                            =&\frac{112}{125}\\
                            =&0.896\\
                    \end{split}
                \end{equation*}
        \subsection{}
            \subsubsection{}
                \begin{equation*}
                    \begin{split}
                        P_{Undetected}=&C_n^0\times0.2^n\\
                            =&0.2^n\\
                            =&0.0001\\
                        0.2^n=&0.0001\\
                        n=&\log_{0.2}0.0001\\
                        n\approx&5.7\\
                        n=&6\\
                    \end{split}
                \end{equation*}
            \subsubsection{}
                \begin{equation*}
                    \begin{split}
                        P_{Undetected}=&C_3^0\times p^n\\
                            =&p^3\\
                            =&0.0001\\
                        p^3=&0.0001\\
                        p\approx&0.0464\\
                        1-p\approx&0.9536
                    \end{split}
                \end{equation*}
    \section{}
        \subsection{}
            \begin{equation*}
                \begin{split}
                    P_a=&C_{15}^50.05^5\times0.95^{10}\\
                        \approx&0.0006\\
                \end{split}
            \end{equation*}
        \subsection{}
            \paragraph{
                My recation: WTF??? I'm so unfortunate.
            }
\end{document}
\documentclass[titlepage]{article}
\title{Notes of Probability, MATH 2005}
\author{Box, ZHANG Huakang}
\usepackage{setspace}
\usepackage{amsmath}
\usepackage{amssymb}
\usepackage{amsthm}
\usepackage{graphicx} 
\usepackage{float} 
\usepackage{fancyhdr}                                
\usepackage{lastpage}          
\usepackage{textcomp}                               
\usepackage{layout}   
\usepackage{subfigure} 
\newtheorem*{remark}{Remark}

\begin{document}
    \maketitle
    \tableofcontents
    \section{Probability Measure}
        \paragraph{
            A probability measure must datisfy the following three postulates:
        }
        \begin{enumerate}
            \item $\mathbb{P}(S)=1$
            \item For each event $A$, the probability of $A$ is a nonnegative real number, i.e., $\mathbb{P}(A)\geq 0$
            \item if $\{A_n\}$ is an infty sequence of events if $F$ such that, for any $i\neq j$, $A_i\cap A_j=\emptyset$, then $$\mathbb{P}(\cup_{n=1}^\infty)=\sum_{n=1}^\infty\mathbb{P}(A_n)$$
        \end{enumerate}
    \section{Conditional Probability}
        \paragraph{
            Let $(S,F,\mathbb{P})$ be a probability space, and let $A$ and $B$ are two random events in the sample space $S$ with $\mathbb{P}(B)\neq 0$. Then the Conditional probability of $A$ given $B$ is defined by 
            $$\mathbb{P}(A|B)=\frac{\mathbb{P}(A\cap B)}{\mathbb{P}(B)}$$
            $$\mathbb{P}(A\cap B)=\mathbb{P}(A|B)\mathbb{P}(B)=\mathbb{P}(B|A)\mathbb{P}(A)$$
        }
        \paragraph{
            Let $n$ random events $B_1,...B_n$ constitute a partition of the sample space $S$ and satisfy that $\mathbb{P}(B_k)\neq 0$ for each $k=1,...,n$. Then, for any random event $A$,
            $$\mathbb{P}(A)=\sum_{k=1}^n\mathbb{P}(B_k)\mathbb{P}(A|B_k)$$
        }
    \section{Bayes' Theorem}
        \paragraph{
            Let $(S,F,\mathbb{P})$ be a probability space,and let $n$ random events $B_1,...B_n$ constitute a partition of the sample space $S$ and satisfy that $\mathbb{P}(B_k)\neq 0$ for each $k=1,...,n$. Then, for any random event $A$ with $\mathbb{A}\neq 0$ and for each $B_k$,
            $$\mathbb{P}(B_k|A)=\frac{\mathbb{P}(B_k)\mathbb{P}(A|B_k)}{\sum_{j=1}^n\mathbb{P}(B_j)\mathbb{P}(A|B_j)}$$
        }
    \section{Independent Events}
        \paragraph{$$\mathbb{P}(A\cap B)=\mathbb{P}(A)\mathbb{P}(B)$$}
        $$f(x,y)=g(x)h(y)$$
    \section{Basic Properties of Expected Values}
        \paragraph{
            $$\mathbb{E}[aX+b]=a\mathbb{E}[X]+b$$
            $$\mathbb{E}[\sum_{k=1}^n c_k\phi_k(X)]=\sum_{k=1}^nc_k\mathbb{E}[\phi_k(X)]$$
        }
    \section{Moment of Random Variables}
        \paragraph{
            The $r$-th moment of $X$
            $$\mu'_r=\mathbb{E}[X^r]$$
            $$\mu'_1=\mu$$
            The $r$-th central moment of $X$
            $$\mu_r=\mathbb{E}[(X-\mu)^r]$$
            $$\mu_2=var(X)=\mathbb{E}[X^2]-\mu^2$$
        }
    \section{Basic Properties of Variance}
        \paragraph{
            $$var(aX+b)=a^2var(X)$$
        }
        \subsection{Chebyshevs inequality}
            \paragraph{
                $$0\leq\mathbb{P}(|X-\mu|\geq \epsilon)<\frac{\sigma^2}{\epsilon^2}$$
                or
                $$1\geq \mathbb{P}(|X-\mu|<k\sigma)\geq1-\frac{1}{k^2}$$
            }
    \section{Product Moments of Random Variables}
        \paragraph{
            $$\mathbb{E}[\phi(X,Y)]=\sum_x \sum_y \phi(x,y)f(x,y)$$
            $$\mathbb{E}[\phi(X,Y)]=\int_{-\infty}^\infty \int_{-\infty}^\infty \phi(x,y)f(x,y)dxdy$$
            If $X$ and $Y$ are independent,
            $$\mathbb{E}(XY)=\mathbb{E}(X)\mathbb{E}(Y)$$
        }
        \subsection{$r$th and $s$th product moment about origin}
            $$\mu'_{r,s}=\mathbb{E}(X^rY^s)$$
        \subsection{$r$th and $s$th product moment about mean}
            $$\mu_{r,s}=\mathbb{E}((X-\mu_X)^r(Y-\mu_Y)^s)$$
        \subsection{Covariance}
            \paragraph{
                $$cov(X,Y)=\mathbb{E}[(X-\mu_Y)(Y-\mu_Y)]=\mathbb{E}[XY]-\mathbb{E}[X]\mathbb{E}[Y]$$
                If $X$ and $Y$ are independent,
                $$cov(X,Y)=0$$
            }
    \section{Properties of Mean and Variance}
        \paragraph{
            $$\mathbb{E}[X+Y]=\mathbb{E}[X]+\mathbb{E}[Y]$$
            $$var(X+Y)=var(X)+var(Y)+cov(X,Y)$$
        }
    \section{Conditional Expectation}
        \paragraph{
            $$\mathbb{E}[X|Y=y]=\sum_x\phi(x)f(x|y)=\int_{-\infty}^\infty \phi(x)f(x|y)dx$$
        }
    \section{Discrete Uniform Distributions}
        \subsection*{Definition}
            A discrete random variable $X$ is said to have a \textit{discrete uniform distribution}, and it is called a \textit{discrete uniform variable}, if it can take on $k$ different values: $x_1,x_2,...,x_k$, and its probability distribution $f(x)$ is given by
            $$f(x_i)=\frac{1}{k}$$
            where $i=1,2,...,k$.

        \subsection*{Mean and Variance}
            \paragraph{
                \begin{equation*}
                    \begin{split}
                        \mathbb{E}[X]=&\sum_{i=1}^k x_if(x_i)\\
                        =&\frac{1}{k}\sum x_i\\
                        \mathbb{E}[X^2]=&\sum_{i=1}^k {x_i}^2f(x_i)\\
                        =&\frac{1}{k}\sum x_i^2\\
                        var(X)=&\mathbb{E}[X^2]-(\mathbb{E}[X])^2\\
                        =&\frac{1}{k}\sum x_i^2-(\frac{1}{k}\sum x_i)^2\\
                    \end{split}
                \end{equation*}
            }

    \section{Bernoulli Distributions}
        \subsection*{Definition}
            Support that and experiment has two possible outcomes: success and failure, and their probability are respectively, $\theta$ and $1-\theta$. Then, this experiment is called a \textit{Bernoulli Distributions}. Let $X$ be the number of successes of a Bernoulli experiment, i.e. $X=1$ or $X=0$. Then, $X$ is called a random variable having the Bernoulli probability distribution, which is given by
            $$f(x;\theta)=\theta^x(1-\theta)^{1-x}$$
            where $x=0,1$ and $0<\theta<1$ is a parameter.

        \subsection*{Mean and Variance}
            \paragraph{
                \begin{equation*}
                    \begin{split}
                        \mathbb{E}[X]=&\theta\\
                        \mathbb{E}[X^2]=&\theta\\
                        var(X)=&\mathbb{E}[X^2]-(\mathbb{E}[X])^2\\
                        =&\theta-\theta^2\\
                        =&\theta(1-\theta)
                    \end{split}
                \end{equation*}
            }
    \section{Binomial Distributions}
        \subsection*{Definition}
            Let $n$ be a nutural number, and let $0<\theta<1$. Then, a discrete random variable $X$ is said to have a \textit{binomial distribution}, and $X$ is called a binomial random variable, if its probability distribution $b(x;n,\theta)$ is given by
            $$b(x;n,\theta)=C_n^x\theta^x(1-\theta)^{n-x}$$
            where $x=1,2,...,n$ ,$n$ and $\theta$ are two parameters, and 
            $$C_n^x=\frac{n!}{x!(n-x)!}$$
            is the total number of combinations of $n$ distinct numbers taken $x$ numbers at a time.
        \subsection*{Remark}
        We consider $n$ independent Bernoullis experiments, in
        which the parameter $\theta$ (the probability of a success) is the same
        for each experiment. Let $X$ be the total number of successes in
        this sequence of $n$ independent Bernoullis experiments. Then, we
        can see that $X$ is a random variable having a binomial distribution
        with parameters $n$ and $\theta$, i.e., we have the following result.
        \\
        Let $X_1,X_2,...,X_n$ be $n$ independent Bernoulli random variables with the same parameter $\theta$. Then, the random variable $X=X_1+X_2+...+X_n$ has a binomial distribution with parameters $n$ and $\theta$.
        \subsection*{Mean and Variance}
                \begin{equation*}
                    \begin{split}
                        \mathbb{E}[X]=&n\theta\\
                        var(X)=&var(x_1+X_2+...+X_n)\\
                            =&var(X_1+X_2+...+X_{n-1})+var(X_n)-2cov(X_1+X_2+...+X_{n-1},X_n)\\
                            &...\\
                            =&var(X_1)+var(X_2)+...+var(X_n)\\
                            =&n\theta(1-\theta)
                    \end{split}
                \end{equation*}
            
        \subsection*{Theorem}
                    \begin{equation*}
                        \begin{split}
                            b(x;n,\theta)=&C_n^x\theta^x(1-\theta)^{n-x}\\
                                =&C_n^{n-x}(1-\theta)^{1-\theta}\theta^x\\
                                =&b(n-x;n,1-\theta)\\
                        \end{split}
                    \end{equation*}
                    Since a binomial random variable $X$ with parameters $n$ and $\theta$ is the total number of successes in $n$ independent Bernoullis experiments. $Y=\frac{X}{n}$ is the proportion of successes in $n$ independent Bernoullis experiments.
                    $$\mathbb{E}[Y]=\theta$$
                    $$var(Y)=\frac{\theta(1-\theta)}{n}$$
                

    \section{Negative Binomial Distributions}
        \subsection*{Definition}
                Let $k$ be a nutural number and let $0<\theta<1$. Then, a discrete random variable $Y$ is said ti have a (Pascal) negative binomial distribution, and it is called a (Pascal) negative binomial random variable, if its probability distribution $b^*(y;k,\theta)$ is given by
                $$b^*(y;k,\theta)=C_{y-1}^{k-1}\theta^k(1-\theta)^{y-k}$$
                where $k$ and $\theta$ are two parameters.

        \subsection*{Mean and Variance}
            \begin{equation*}
                \begin{split}
                    \mathbb{E}[Y]=&\sum_{i=k}^\infty ib^*(i;k.\theta)\\
                        =&\sum_{i=k}^\infty iC_{i-1}^{k-1}\theta^k(1-\theta)^{i-k}\\
                        &...\\
                        =&\frac{k}{\theta}\\
                    var(Y)=&\frac{k}{\theta}(\frac{1}{\theta}-1)
                \end{split}
            \end{equation*}

        \subsection*{Theorem}
            Let $Y$ be a negative binomial random variable with parameters $k$ and $\theta$. Then for each $y=k,k+1...$,
            $$b^*(y;k,\theta)=\frac{k}{y}b(k;y,\theta)$$
            \begin{proof}
                By the definition, we have 
                \begin{equation*}
                    \begin{split}
                        b^*(y;k,\theta)=&C_{y-1}^{k-1}\theta^k(1-\theta)^{y-k}\\
                            =&\frac{(y-1)!}{(k-1)!(y-k)!}\theta^k(1-\theta)^{y-k}\\
                            =&\frac{k}{y}\frac{y!}{k!(y-k)!}\theta^k(1-\theta)^{y-k}\\
                            =&\frac{k}{y}b(k;y,\theta)
                    \end{split}
                \end{equation*}
            \end{proof}
    \section{Geometric Distributions}
        \subsection*{Definition}
            If $X$ is a (Pascal) negative binomial random variable with parameters $k=1$ and $\theta$, we say that this random variable $X$ has a geometric distribution, and we also call this random variable as a geometric random variable. By the definition of negative binomial distribution, we see that the probability distribution $g(x;\theta)=b^*(x;1,\theta)$ of geometric distribution is given by 
            $$g(x;\theta)=\theta(1-\theta)^{x-1}$$
            where $\theta$ is a parameter.
        \subsection*{Mean and Variance}
            \begin{equation*}
                \begin{split}
                    \mathbb{E}[X]=&\frac{1}{\theta}\\
                    var(X)=&\frac{1}{\theta}(\frac{1}{\theta}-1)
                \end{split}
            \end{equation*}
            % \begin{proof}
            %     For any $0<\theta<1$, Since $0<1-\theta<1$ by the geometric series we have the following identity
            %     $$\sum_{x=1}^\infty \theta(1-\theta)^{x-1}=\theta\frac{1}{1-(1-\theta)}=1$$

            % \end{proof}
        \subsection*{Theorem}
            \paragraph{
                The geometric distribution has the memoryless property, i.e. if $X$ is a geometric random variable, then, for any nature $n$,
                $$\mathbb{P}(X=x+n|X>n)=\mathbb{P}(X=x)$$
            }
    \section{Hyper-geometric Distributions}
        \subsection*{Definition}
            \paragraph{
                A random variable $X$ is said to have a hyper-geometric distribution, and it is referred to as a hyper-geometric random variable, if its probability distribution is given by 
                $$h(x;n,N,k)=\frac{C_k^xC_{N-k}^{n-x}}{C_N^n}$$
                for $x=0,1,...,n$ with $x\leq k$ and $n-x\leq N-k$, where $n,N,k$ are parameters.
            }
        \subsection*{Mean and Variance}  
            \begin{equation*}
                \begin{split}
                    \mathbb{E}[X]=&\frac{nk}{N}\\
                    var(X)=&\frac{nk(N-k)(N-n)}{N^2(N-1)}\\
                \end{split}
            \end{equation*}

    \section{Possion Distributions}
        \subsection*{Definition}
            \paragraph{
                A discrete random variable $X$ is said to have a Possion distribution, and it is referred to as a Possion random variable, if its probability distribution is given by
                $$p(x;\lambda)=\frac{\lambda^x}{x!}e^{-\lambda}$$
            }
        \subsection*{Mean and Variance}  
            \paragraph{
                \begin{equation*}
                    \begin{split}
                        \mathbb{E}[X]=&\lambda\\
                        var(X)=&\lambda
                    \end{split}
                \end{equation*}
            }
    \section{Multivariate Distributions}
        \subsection{Polynomial Distributions}
            \subsubsection*{Definition}
                \paragraph{
                    The $k$ random variable $X_1,X_2,...,X_k$ are said to have a polynomial distribution, and they are referred to as polynomial random variable, if their joint probability distribution is given by
                    $$f(x_1,...,x_k;\theta_1,...,\theta_k)=\frac{n!}{x_1!...x_k!}\theta_1^{x_1}...\theta_k^{x_k}$$
                    for $x_i=0,1,...,n$ and $0<\theta_i<1$ for each $i=1,...,k$, where 
                    $$n=\sum_{i=1}^kx_i$$
                    $$\sum_{x=1}^k\theta_i=1$$
                }
        \subsection{Multivariate Hyper-geometric Distributions}
            \subsubsection*{Definition}
                \paragraph{
                    The $k$ random variable $X_1,...,X_k$ are said to have a multivariate hyper-geometric distribution, and ther are referred to as multivariate hyper-geometric random variable, if their joint probability distribution is given by
                    $$f(x_1,...,x_k;n,j_1,...,j_k)=\frac{C_{j_1}^{x_1}...C_{j_k}^{x_k}}{C_N^n}$$
                    for $x_i=0,1,...,n$ with $x_i\leq j_i$ for each $i=1,...,k$, where 
                    $$n=\sum_{i=1}^kx_i$$
                    $$N=\sum_{i=1}^kj_i$$
                }

    \section{Uniform Densities}
        \subsection*{Definition}
            \paragraph{
                A continuous random variable $X$ is said to have a uniform density withe parameters $\alpha$ and $\beta$, and it is referred to as a continuous uniform random variable, if its probability density is given bu
                $$f(x)=\frac{1}{\beta-\alpha}$$ 
                for $\alpha<x<\beta$, and $f(x)=0$ elsewhere, where $\alpha$ and $\beta$ are two parameters with $\alpha<\beta$.
            }
        \subsection*{Mean and Variance}
            \begin{equation*}
                \begin{split}
                    \mathbb{E}[X]=&\frac{\alpha+\beta}{2}\\
                    var(X)=&\frac{(\beta-\alpha)^2}{12}\\
                \end{split}
            \end{equation*}
    \section{The Gamma Function}
        \subsection*{Definition}
            \paragraph{
                The gamma function $\Gamma(x)$ is a real function which is defined on $(0,\infty)$ and is given by
                $$\Gamma(x)=\int_0^\infty y^{x-1}e^{-y}dy$$
                for each $x\in (0,\infty)$
            }
        \subsection*{Theorem}
            \paragraph{
                $$\Gamma(n)=(n-1d)!$$
                for each natural number $n$.
                $$\Gamma(\frac{1}{2})=\sqrt{\pi}$$
            }
    \section{The Beta Function}
        \subsection*{Definition}
            \paragraph{
                The beta function $B(u,v)$ is a real bivariate function, which is defined for each $(u,v)\in (0,\infty)\times(0,\infty)$, and is given by 
                $$B(u,v)=\int_0^1t^{u-1}(1-t)^{v-1}dt$$
            }
        \subsection*{Theorem}
            \paragraph{
                The gamma function and the beat function satisfy the following equation
                $$B(u,v)=\frac{\Gamma(u)\Gamma(v)}{\Gamma(u+v)}$$
                for any $(u,v)\in (0,\infty)\times(0,\infty)$
            }
            
    \section{Gamma Distributions}
        \subsection*{Definition}
            \paragraph{
                A random variable $X$ is said to have a gamma distribution, and it is referred to as a gamma random variable, if its probability density is given by 
                $$f(x)=\frac{1}{\beta^{\alpha}\Gamma(\alpha)}x^{\alpha-1}e^{-\frac{x}{\beta}}$$
                for $x>0$, and $f(x)=0$ for $x\leq 0$, where $\alpha>0$ and $\beta>0$ are two parameters.
            }
        \subsection*{Theorem}
            \paragraph{
                The $r$-th moment of a gamma distribution with parameters $\alpha$ and $\beta$ is given by 
                $$\mu'_r=\frac{\beta^r\Gamma(\alpha+r)}{\Gamma(\alpha)}$$
            }
        \subsection*{Mean and Variance}
            \begin{equation*}
                \begin{split}
                    \mathbb{E}[X]=&\alpha\beta\\
                    var(X)=&\alpha\beta^2\\
                \end{split}
            \end{equation*}
    \section{Beta Distributions}
        \subsection*{Definition}
            \paragraph{
                A random variable $X$ is said to have a beta distribution, and it is referred to as a beta random variable, if its probability density is given by
                $$f(x)=\frac{\Gamma(\alpha+\beta)}{\Gamma(\alpha)\Gamma(\beta)}x^{\alpha-1}(1-x)^{\beta-1}$$
            }
        \subsection*{Theorem}
            \paragraph{
                The $r$-th moment of a beta distribution with parameters $\alpha$ and $\beta$ is given by 
                $$\mu'_r=\frac{\Gamma(\alpha+\beta)}{\Gamma(\alpha)}\frac{\Gamma(\alpha+r)}{\Gamma(\alpha+\beta
                +r)}$$
            }
        \subsection*{Mean and Variance}
            \begin{equation*}
                \begin{split}
                    \mathbb{E}[X]=&\frac{\alpha}{\alpha+\beta}\\
                    var(X)=&\frac{\alpha\beta}{(\alpha+\beta)^2(\alpha+\beta+1)}\\
                \end{split}
            \end{equation*}
    \section{Exponential Distributions}
        \subsection*{Definition}
            \paragraph{
                Let $X$ be a gamma random variable with parameters $\alpha$ and $\beta$. If $\alpha=1$ and $\beta=\theta$, this random variable $X$ is said to have an exponential distribution, and it is referred to as an exponential random variable. The density of exponential distribution is given by 
                $$f(x)=\frac{1}{\theta}e^{-\frac{x}{\theta}}$$
                for $x>0$, and $f(x)=0$ for $x\leq 0$
            }
        \subsection*{Theorem}
            \paragraph{
                The exponential distribution satisfies the memoryless property, i.e., if $X$ is an exponential random variable, then, for each $t>0$,
                $$\mathbb{P}(X\geq x+t|X\geq t)=\mathbb{P}(X\geq x)$$
            }
        \subsection*{Mean and Variance}
            \begin{equation*}
                \begin{split}
                    \mathbb{E}[X]=&\theta\\
                    var(X)=&\theta^2\\
                \end{split}
            \end{equation*}
    \section{Normal Distributions}
        \subsection{The Standard Normal Distributions}
            \subsubsection*{Definition}
                \paragraph{
                    A random variable $Z$ is said to have the standard normal distribution, and so that it is called a standard normal distribution variable and it is denoted by $Z\sim N(0,1)$, if its density is given by 
                    $$\phi(z)=\frac{1}{\sqrt{2\pi}}e^{-\frac{1}{2}z^2}$$
                } 
        \subsection{Normal Distributions}
            \subsubsection*{Definition}
                \paragraph{
                    Let $X$ be a normal random variable with parameters $\mu$ and $\sigma>0$, and let $Z=\frac{X-\mu}{\sigma}$. Then, $Z$ is a standard normal random variable, i.e., $Z\sim N(0,1)$
                }
            \subsubsection*{Mean and Variance}
                \begin{equation*}
                    \begin{split}
                        \mathbb{E}[X]=&\mu\\
                        var(X)=&\sigma^2\\
                    \end{split}
                \end{equation*}
        \subsection{The Normal Approximation}
            \paragraph{
                Let $X_n$ be a random variable having a binomial distribution with parameters $n$ and $\theta$, and let $$Z_n=\frac{X_n-n\theta}{\sqrt{n\theta(1-\theta)}}$$
            }
    \section{Bivariate Normal Distributions}
        \subsection*{Definition}
            \paragraph{
                A pair of random variables $(X_1,X_2)$ is said to have a bivariate normal distribution, and $X_1$ and $X_2$ are referred to as jointly normal distributed random variables, if their joint density is given by 
                $$\Phi(x_1,x_2)=\frac{1}{2\pi\sigma_1\sigma_2\sqrt{1-\rho^2}}e^{-\psi(x_1,x_2)}$$
                for $(x_1,x_2)\in \mathbb{R}^2$
                where
                $$\psi(x_1,x_2)=\frac{1}{2(1-\rho^2)}(\frac{(x_1-\mu_1)^2}{\sigma_1^2}-2\rho\frac{(x_1-\mu_1)(x_2-\mu_2)}{\sigma_1\sigma_2}+\frac{(x_2-\mu_2)^2}{\sigma_1^2})$$and $\mu_i,\sigma_i>0$ with $i=1,1$, and $|\rho|<1$ are all parameters.
            }
        \subsection*{Theorem}
            \paragraph{
                The marginal densities are respectively given by 
                $$\phi_1(x_1)=\frac{1}{\sigma_1\sqrt{2\pi}}e^{-\frac{(x_1-\mu_1)^2}{2\sigma_1^2}}$$
                $$\phi_2(x_2)=\frac{1}{\sigma_2\sqrt{2\pi}}e^{-\frac{(x_2-\mu_2)^2}{2\sigma_2^2}}$$
                The covariance of $X_1$ and $X_2$ is given by 
                $$cov(X_1,X_2)=\rho\sigma_1\sigma_2$$    
                $X_1$ and $X_2$ are independent,if and only if $\rho=0$
            }
    \section{The Distribution Function Technique}
            \paragraph{
                Let $X$ be a random variable with the distribution function $F(x)$, and let $Y=u(X)$, where $u(x)$ is an increasing function such that its inverse function $x=u^{-1}(y)$ exists. Then the distribution function of $Y$ is given by 
                $$G(y)=F(u^{-1}(y))$$
                for all real numbers $y$ in the range of $Y$.
            }
    \section{The Transformation Technique}
        \paragraph{
            Let $X$ be a random variable whose density function is $f(x)$, and let $y=u(x)$ is a differentiable function such that its inverse function $x=u^{-1}(y)$ exists. Then, the density function $g(y)$ of $Y=u(X)$ is given by 
            $$g(y)=f(w(y))|w'(y)|$$ when $w'(y)\neq 0$ 
            and $g(y)=0$ elsewhere where $w(y)=u^{-1}(y)$.
        }
    \section{Moment Generating Functions}
        \subsection*{Definition}
            \paragraph{
                Let $X$ be a random variable. The moment-generating dunction of $X$ id defined by 
                $$M_X(t)=\mathbb{E}[e^{tX}]$$
                for each real number $t$ in which the expectation exists, i.e., when $X$ is continuous with density $f(x)$.
                $$M_X(t)=\int_{-\infty}^\infty e^{tx}f(x)dx$$
                when $X$ is discrete with probability distribution $f(x)$,
                $$M_X(t)=\sum_xe^{tx}f(x)$$
            }
        \subsection{Some Moment-generating functions}
        \subsubsection{Binomial random variable}
            \paragraph{
                $$M_X(t)=(1+\theta(e^t-1))^n$$
            }
        \subsubsection{Possion random variable}
            \paragraph{
                $$M_X(t)=e^{\lambda(e^t-1)}$$
            }
        \subsubsection{Gamma random variable}
            \paragraph{
                $$M_X(t)=(1-\beta t)^{-\alpha}$$
            }
        \subsubsection{Exponential random variable}
            \paragraph{
                $$M_X(t)=\frac{1}{1-\theta t}$$
            }
        \subsubsection{Chi-square random variable}
            \paragraph{
                $$M_X(t)=(1-2t)^{-\frac{v}{2}}$$
            }
        \subsubsection{Normal random variable}
            \paragraph{
                $$M_X(t)=e^{\mu t+\frac{1}{2}\sigma^2 t^2}$$
            }
        \subsection{Properties of Moment Generating Functions}
            \paragraph{
                $$\frac{d^kM_X(t)}{dt^k}|_{t=0}=\mu'_k=\mathbb{E}[X^k]$$
            }
            \paragraph{
                $$M_{aX+b}(t)=e^{tb}M_X(at)$$
            }
            \paragraph{
                Let $X$ and $Y$ be two independent random variables,
                $$M_{X+Y}(t)=M_X(t)M_Y(t)$$
            }
    \section{Moment Generating Function Technique}
        \subsection{Possion Distribution}
            \paragraph{
                Let $X_1$ and $X_2$ be two independent random variables, and let $X_1$ and $X_2$ have the Possion distributions $p(x_1,\lambda_1)$ and $p(x_2,\lambda_2)$, respectively. Then, the random variable $Y=X-1+X_2$ ahs a Possion distribution $$p(y,\lambda_1+\lambda_2)$$
            }
        \subsection{Normal Distribution}
            \paragraph{
                Let $X_1$ and $X_2$ be two independent random variables such that $X_i\sim N(\mu_i,\sigma_i)$, $i=1,2$, respectively, and let $a$ and $b$ be two constants such that $a^2+b^2\neq 0$. then, the random variable $Y=aX_1+bX_2$ has a normal distribution with mean $\mu=a\mu_1+b\mu_2$ and variance $\sigma^2=a^2\sigma_1^2+b^2\sigma_2^2$.
            }
        \subsection{Exponential Distribution}
            \paragraph{
                Let $X_1$ and $X_2$ be two independent random variables having an exponential distributions with the same parameter $\theta$, respectively. Then the random variable $Y=X_1+X_2$ has a gamma distribution with $\alpha=2$ and $\beta=\theta$
            }
\end{document}   

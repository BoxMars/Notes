\documentclass{article}
\usepackage{setspace}
\usepackage{amsmath}
\usepackage{amssymb}
\usepackage{amsthm}
\usepackage{graphicx} 
\usepackage{float} 
\usepackage{fancyhdr}                                
\usepackage{lastpage}                                           
\usepackage{layout}   
\usepackage{subfigure} 
\pagestyle{fancy}  
\lhead{ZHANG HUAKANG}
\chead{Assignment 5} 
\rhead{DB92760} 
\renewcommand{\baselinestretch}{1.05}
\title{Assignment 4 of MATH 2005}
\author{ZHANG Huakang/DB92760}

\begin{document}
    \maketitle
    \section{}
    \paragraph{
        \begin{equation*}
            \begin{split}
                E[X]=&\sum_{n=1} ^knf(n)\\
                    =&\frac{1}{k}\sum_{n=1}^k n\\
                    =&\frac{1}{k}\ \frac{k(k+1)}{2}\\
                    =&\frac{k+1}{2}\\
                var(X)=&\sum_{n=1}^k (n-E[X])^2f(n)\\
                    =&\frac{1}{k}\sum_{n=1}^k(n-\frac{k+1}{2})^2\\
                    =&\frac{1}{k}\sum_{n=1}^k(n^2-n(k+1)-\frac{(k+1)^2}{4})\\
                    =&\frac{1}{k}(\frac{k(k+1)(2k+1)}{6}-\frac{k(k+1)}{2}(k+1)+\frac{k(k+1)^2}{4})\\
                    =&\frac{(k+1)(2k+1)}{6}-\frac{(k+1)^2}{2}+\frac{(k+1)^2}{4}\\
                    =&\frac{(k+1)(2k+1)}{6}-\frac{(k+1)^2}{4}\\
                    =&(k+1)\frac{k-1}{12}\\
                    =&\frac{k^2-1}{12}
            \end{split}
        \end{equation*}
    }

    \section{}
    \paragraph{
        \begin{equation*}
            \begin{split}
                b(x;n,\theta)=&C_n^x\theta^x(1-\theta)^{n-x}\\
                    =& C_{n}^{n-x}(1-\theta)^{n-x}\theta^x\\
                    =&b(n-x;n,1-\theta)\\
            \end{split}
        \end{equation*}
    }
    \subsection{}
        \paragraph{
            \begin{equation*}
                \begin{split}
                    B(n-x;n,1-\theta)-B(n-x-1;n,1-\theta)=&\sum_{y=0}^{n-x}b(y;n,1-\theta)\\
                        &-\sum_{y=0}^{n-x-1}b(y;n,1-\theta)\\
                        =&b(n-x;n,1-\theta)\\
                        =&b(x;n,\theta)
                \end{split}
            \end{equation*}
        }
    \subsection{}
        \paragraph{
            $$B(n;n,1-\theta)=\sum_{y=0}^nb(y;n,\theta)=1$$
            \begin{equation*}
                \begin{split}
                    B(x;n,\theta)=&\sum_{y=0}^xb(y;n,\theta)\\
                        =&\sum_{y=0}^x[B(n-y;n,1-\theta)-B(n-y-1;n,1-\theta)]\\
                        =&B(n;n,1-\theta)+(B(n-1;n,1-\theta)-B(n-1;n,1-\theta)...)-B(n-x-1;n,1-\theta)\\
                        =&B(n;n,1-\theta)-B(n-x-1;n,1-\theta)\\
                        =&1-B(n-x-1;n,1-\theta)\\
                \end{split}
            \end{equation*}
        }
    \section{}
        \begin{proof}
            \begin{equation*}
                \begin{split}
                    b(x;n,\theta)=&C_n^x\theta^x(1-\theta)^{n-x}\\
                        =&\frac{n!}{x!(n-x)!}\theta^x(1-\theta)^{n-x}\\
                        \\
                    b(x+1;n,\theta)=&C_n^{x+1}\theta^{x+1}(1-\theta)^{n-x-1}\\
                        =&\frac{n!}{(x+1)!(n-x-1)!}\theta^{x+1}(1-\theta)^{n-x-1}\\
                        \\
                    \frac{b(x;n,\theta)}{b(x+1;n,\theta)}=& \frac{x+1}{n-x}\frac{1-\theta}{\theta}\\
                        =&\frac{(x+1)(1-\theta)}{\theta(n-x)}\\
                        \\
                    b(x+1;n,\theta)=&\frac{\theta(n-x)}{(x+1)(1-\theta)}b(x;n,\theta)\\
                \end{split}
            \end{equation*}
        \end{proof}         
        \paragraph{
            By the definition, when $\theta=\frac{1}{2}$
            $$b(x;n,\frac{1}{2})=C_n^x(\frac{1}{2})^n$$
            \begin{equation*}
                \begin{split}
                    \frac{b(x;n,\theta)}{b(x+1;n,\theta)}=&\frac{(x+1)(1-\theta)}{\theta(n-x)}\\
                        =&\frac{x+1}{n-x}
                \end{split}
            \end{equation*}
            When 
            $$\frac{x+1}{n-x}>1$$
            we can get 
            $$x>\frac{n-1}{2}$$
        }
        \subsection*{a}
        \paragraph{
            $n$ is an even number and $x \in \mathbb{N}$.
            Thus when
            $x\geq\frac{n}{2}$
            $$b(x;n,\theta)>b(x+1;n,\theta)$$
            Similarily, when $x\leq\frac{n}{2}$
            $$b(x;n,\theta)<b(x+1;n,\theta)$$
            Therefore, we can get a maximum at $x=\frac{n}{2}$
        }
        \subsection*{b}
        \paragraph{
            $n$ is an odd number and $x \in \mathbb{N}$.
            Thus when $x\geq\frac{n-1}{2}$
            $$b(x;n,\theta)\geq b(x+1;n,\theta)$$
            When $\frac{x+1}{n-x}=1$,i.e. $x=\frac{n-1}{2}$
            which means $$b(\frac{n-1}{2};n,\theta)=b(\frac{n+1}{2};n,\theta)$$
            We can also get that 
            $$b(x;n,\theta)\leq b(x+1;n,\theta)$$
            when $x<\frac{n-1}{2}$
            Therefore, we can get a maximum at $x=\frac{n-1}{2}$ or $x=\frac{n+1}{2}$
        }


\end{document}
\documentclass{article}
\usepackage{setspace}
\usepackage{amsmath}
\usepackage{amssymb}
\usepackage{amsthm}
\usepackage{graphicx} 
\usepackage{float} 
\usepackage{fancyhdr}                                
\usepackage{lastpage}                                           
\usepackage{layout}   
\usepackage{subfigure} 
\pagestyle{fancy}  
\lhead{ZHANG HUAKANG}
\chead{Assignment 7} 
\rhead{DB92760} 
\renewcommand{\baselinestretch}{1.05}
\title{Assignment 7 of MATH 2003}
\author{ZHANG Huakang/DB92760}

\begin{document}
    \maketitle

    \section{}
        \paragraph{
            $\forall x\in \mathbb{R}$, we can find a sequence $(x_n)$, s.t. $x_n\in \mathbb{Q},\forall n\in\mathbb{N}$ and $\lim (x_n)=x$
            \begin{equation*}
                \begin{split}
                    f(x)=&\lim f(x_n)\\
                        =&\lim g(x_n)\\
                        =&g(x)
                \end{split}
            \end{equation*}
            since $f$ and $g$ are continuous.
        }
    
    \section{}
        \begin{proof}
            Let $a,b$ be real number s.t. $a<b$, we get
            \begin{equation*}
                \begin{split}
                    &0<\frac{1}{n}<b-a\\
                    \rightarrow&0<\frac{1}{2^n}<\frac{1}{n}<b-a\\
                    \rightarrow&1<2^n(b-a)=b2^n-a2^n\\
                \end{split}
            \end{equation*}
            by Archimedean Property where $n\in\mathbb{N}$
            Thus,$\exists m\in \mathbb{Z}$ s.t.
            $$a2^n<m<b2^n$$
            i.e.
            $$a<\frac{m}{2^n}<b$$
            Thus, $\{\frac{m}{2^n}\}$ is dense in $\mathbb{R}$.
            Let $x\in \mathbb{R}$ is arbitrary. We can find a sequence$(x_n)$ s.t
            $x_n=\frac{i}{2^j},i,j\in \mathbb{N}$ and $$x=\lim x_n$$ 
            \begin{equation*}
                \begin{split}
                    h(x)=&\lim h(x_n)\\
                        =&\lim h(\frac{i}{2^j})\\
                        =&0
                \end{split}
            \end{equation*}
        \end{proof}

    \section{}
        \begin{proof}
            By the definition, let $\epsilon=\frac{f(c)}{2}$, $\exists \delta>0$, s.t. $\forall x\in\mathbb{R}, |x-c|<\delta$
            \begin{equation*}
                \begin{split}
                    &|f(x)-f(c)|<\epsilon\\
                    \rightarrow&-\epsilon<f(x)-f(c)\\
                    \rightarrow&f(c)-\epsilon<f(x)\\
                    \rightarrow&\frac{f(c)}{2}<f(x)\\
                    \rightarrow&0<f(x)
                \end{split}
            \end{equation*}
            Thus we have $$x\in V_\delta(c)\Rightarrow x\in P$$
            i.e.
            $$V_\delta(c)\subset P$$
        \end{proof}
    \section{}
        \begin{proof}
            \begin{equation*}
                \begin{split}
                    (s_n)\subset S \Rightarrow &f(s_n)>g(s_n),\forall n\in \mathbb{N}\\
                    \Rightarrow&\lim f(s_n)>\lim g(s_n)\\
                    \Rightarrow&f(s)>g(s)\\
                \end{split}
            \end{equation*}
        \end{proof}

    \section{}
        \begin{proof}
            Let $(x_n)$ be any sequence s.t.$$\lim x_n=x$$
            \begin{equation*}
                \begin{split}
                    &\Rightarrow\lim (x_n-x)=0\\
                    &\Rightarrow\lim (x_n-x+x_0)=x_0\\
                    &\Rightarrow\lim (f(x_n-x+x_o))=f(x_0)\\
                    &\Rightarrow\lim (f(x_n-x)+f(x_0))=f(x_0)\\
                    &\Rightarrow\lim f(x_n-x)=0\\
                    &\Rightarrow\lim (f(x_n-x)+f(x))=f(x)\\
                    &\Rightarrow\lim f(x_n-x+x)=f(x)\\
                    &\Rightarrow\lim f(x_0)=f(x)\\
                \end{split}
            \end{equation*}
            $f$ is continuous.
        \end{proof}

    \section{}
        \begin{proof}
            By the Max-Min Theorem, $\exists x\in I$ such that
            $$f(x')<f(x)$$
        $\forall x\in I$.
        Since $f(x)>0$, $f(x')>0$. Let $\alpha=f(x')$,
        $$0<\alpha\leq f(x)$$
        \end{proof}

    \section{}
        \begin{proof}       
            Let $(x_n)\in E$ and $\lim x_n=x_0$. Since $(x_n)\in E$, we have $f(x_n)=g(x_n),\forall n\in \mathbb{N}$
            $$\lim g(x_n)=f(x_n)$$
            $$g(\lim x_n)=f(\lim x_n)$$
            $$f(x_0)=g(x_0)$$
            Thus,
            $$x_0\in E$$
        \end{proof}   
        
    \section{}
        \begin{proof}
            $\forall x_1\in I, \exists x_2\in I$, s.t.
            $$|f(x_2)|  \leq \frac{1}{2}f(x_1)$$
            We can find a sequence $(x_n)\in I$ s.t.
            $$|f(x_{n+1})|  \leq \frac{1}{2}f(x_n)<(\frac 1 2)^nf(x_1)$$ 
            By Bolzano-Weirstrass Theorem there exists a subsequence $(x_{p(n)})$ of $(x_n)$ which converges to $c\in I$. Since 
            $$\lim f(x_{p(n)})=0$$ since $|f(x_{n+1})|<(\frac{1}{2})^nf(1)$
            .Thus
            $$f(c=0)$$
        \end{proof}

    \section{}
        \paragraph{
            $$f(-8)=503,f(-7)=-9$$
            $$f(1)=-1,f(2)=63$$
            Thus there exist $c_1\in (-8,-7)$ and $c_2\in (1,2)$ s.t.
            $f(x_1)=f(c_2)=0$
            By calculator, we know $$c_1\approx -7.02$$
            $$c_2\approx 1.03$$
            }
    \section{}
        \paragraph{
            By the property of supermum, there exists a sequence$(x_n)\subset W$ converging to $w$. And $f(x_n)<0$,
            we fet $$\lim f(x_n)=f(w)\leq 0$$
            S0, $w<b$. So $w$ is an interior point of $I$. So if $f(w)<0$ then by the continuity, for some $\delta>0$, $w+\delta\in I$ and $f(w+\delta)<0$ which contradicts the fact that $w$ is supermum of $W$. Thus $$f(w)=0$$ 
        }
\end{document}
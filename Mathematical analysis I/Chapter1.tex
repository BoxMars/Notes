\documentclass{article}
\title{Chapter 1}
\author{Hua Kang}
% \setlength\parindent{24pt}
\usepackage{amsfonts}
\begin{document}
    \maketitle
    \section{Set}
        \subsection{Definition Part}
            \subsubsection{Proper Subset}
                \paragraph{
                    We say that a set $A$ is a \textbf{proper subset} of a set $B$ if $A \subseteq B$, but there is at least one element of $B$ that is not in $A$. In this case we sometimes write
                    $$A \subset B.$$
                    In short, If $A\subseteq B$and $\exists b \in B , b \notin A$, then $A\subset B$.
                    }
                
            
            \subsubsection{Two set is equal}
                \paragraph{
                    If $A\in B$ and $B\in A$, then two set are said to be \textbf{equal}, and we write $A=B$.
                }
                
            
            \subsubsection{Set Operations}
                \paragraph{
                    The \textbf{union} of sets $A$ and $B$ is the set
                $$A \cup B =\{x:x\in A \textbf{or} x\in B\} .$$
                The \textbf{intersection} of the sets $A$ and $B$ is the set
                $$ A\cap B =\{x:x\in A \textbf{and}x\in B\}.$$
                The \textbf{complement of} $B$ \textbf{relative to} $A$ is the set
                $$A\backslash B=\{x:x\in A \textbf{and} x \notin B\}.$$
                }
                

            \subsubsection{Empty set and disjoint}
                \paragraph{
                    The set that has no elements is called the \textbf{empty set} and is denoted by the symbol  $\emptyset$. Two set $A$ and $B$ are sasid to be \textbf{disjoint} if they have no elements in common, this can be expressed by writing $A \cap B =\emptyset$
                }
                
            
            \subsubsection{Infinite union or intersection}

                $$\cup _{n=1} ^{\infty} A_n=\{x:x\in A_n ,\exists n \in \mathbb{N} \}$$

                $$\cap _{n=1} ^{\infty} A_n=\{x:x\in A_n ,\forall n \in \mathbb{N} \}$$
                
        \subsection{Theorem Part}
            \subsubsection{De Morgan Law}
                \paragraph{If $A,B,C$ are sets, then
                $$A \backslash (B\cup C) = (A\backslash B)\cap (A\backslash C)$$
                $$A\backslash (B\cap C) = (A\backslash B)\cup (A\backslash C)$$}
            

        \subsection{Other}

    \section{Function}
        \subsection{Definition Part}
            \subsubsection{Cartesian product}
            \paragraph{If $A$ and $B$ are noempty sets, then the \textbf{Cartesian product} $A \times B$ of $A$ and $B$ is the set of all ordered pairs $(a,b)$ with $a\in A$ and $b\in B$. That is 
            $$A\times B = \{(a,b):a\in A,b\in B\}$$}
                

            \subsubsection{Function}
            \paragraph{Let $A$ and $B$ be setd. Then a \textbf{function} from $A$ to $B$ is a set $f$ of ordered pairs in $A\times B$ such that for each $a\in A$there exists a unique $b\in B$ with $(a,b)\in f$.
            \\
            In other word, if $(a,b)\in f,(a,b')\in f$, then $b=b'$.}
                
            \subsubsection{Domain and Range}
            \paragraph{The set $A$ of first elements of a function $f$ is called the \textbf{domain} of $f$ and is often denoted by $D(f)$
            \\
            The set of all second elements in $f$ is called the \textbf{range} of $f$ and is often denoted by $R(f)$
            \\
            \textbf{Note that}, although $D(f)=A$, we only have $R(f)\subseteq B$
            \textit{B is codomain}
            }

            \subsubsection{Direct and Inverse Images}
            \paragraph{
                Let $f:A\rightarrow B$ be a function with domain $D(f)=A$ and range $R(f)\subseteq B$
                \\
                If $E$ is a subset of $A$, then the \textbf{direct image} of $E$ unser $f$ is the subset $f(E)$ of B given by
                $$ f(E)=\{f(x):x\in E\}$$
                If $H$ is s subset of $B$ , then the \textbf{inverse image} of $H$ under $f$ is the subset $f^{-1}(H)$ of $A$ given by 
                $$ f^{-1}(H)=\{x\in A : f(x)\in H\}$$
            }

            \subsubsection{Injective, surjective and bijective}
            \paragraph{
                The function $f$ is said to be \textbf{injective} (or to be \textbf{one-one}) if whenever $x_1 \neq x_2$, then $f(x_1)\neq f(x_2)$.If $f$ is an injective function, we also say that $f$ is an injective function, we also say that $f$ is an \textbf{ingetion}.
                }
            \paragraph{
                The function $f$ is said to be \textbf{surjective} (or to map $A$ \textbf{onto} $B$) if $f(A)=B$;that is, if the range $R(f)=B$. If $f$ is a surjective function, we also say that $f$ is a \textbf{surjection}.
                }
            \paragraph{
                If $f$ is both inkective and surjective, the $f$ is said to be \textbf{bijective}. If $f$ is bijective, we also say that $f$ is a \textbf{bijetion}
            }

            \subsection{Inversen Function}
            \paragraph{
                If $f:A\rightarrow B$ is a bijection of $A$ onto $B$, then
                $$g:=\{(b,a)\in B\times A :(a,b)\in f\}$$
                is a function on $B$ into $A$. This function is called the \textbf{inverse function} of $f$, and is denoted by $f^{-1}$
            }

            \subsection{Composition of Functions}

            \paragraph{
                If $f:A \rightarrow B$ and $g:B\rightarrow C$, and if $R(f)\subseteq D(g)=B$, then the \textbf{composite function}
                $g\circ f$
            }




        \subsection{Theorem Part}

        \subsection{Other}
        \paragraph{A function $f$ from a set $A$ into a set $B$ is a rule of correspondence that assigns to each element $x$ in $A$ a uniquely determined element $f(x)$ in $B$.
        \\
        The essential condition that :
        \begin{center}
            $(a,b)\in f$ and $(a,b')\in f$ implies that $b=b'$
        \end{center}
        is sometimes called the \textit{vertical line test}.
        \\
        The notation 
        $$f: A \rightarrow B $$
        is often used to indicate that $f$ is a function from $A$ to $B$. We will also say that $f$ is a \textbf{mapping} of $A$ into $B$, or that $f$ \textbf{maps} $A$ into $B$.
        \\
        If $(a,b)$ is an element in$f$, it is customary to write 
        \begin{center}
            $b=f(a)$, or sometimes $a\rightarrow b$.
        \end{center}
        }
        
\end{document}

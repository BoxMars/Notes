\documentclass{article}
\usepackage{setspace}
\usepackage{amsmath}
\usepackage{amssymb}
\usepackage{amsthm}
\usepackage{graphicx} 
\usepackage{float} 
\usepackage{fancyhdr}                                
\usepackage{lastpage}                                           
\usepackage{layout}   
\usepackage{subfigure} 
\pagestyle{fancy}  
\lhead{ZHANG HUAKANG}
\chead{Assignment 5} 
\rhead{DB92760} 
\renewcommand{\baselinestretch}{1.05}
\title{Assignment 5 of MATH 2003}
\author{ZHANG Huakang/DB92760}

\begin{document}
    \maketitle
    \section{}
        \subsection{}
        \paragraph{
            Let $(a_n)=(x_{2n})$ and $(b_n)=(x_{2n-1})$ be the subsequences of $(x_n)$.
        }
        \paragraph{
            We can get the subsequences:
            \begin{equation*}
                \begin{split}
                    a_n=&x_{2n}\\
                        =&1-(-1)^{2n}+\frac{1}{2n}\\
                        =&\frac{1}{2n}\\
                    b_n=&x_{2n-1}\\
                        =&2+\frac{1}{2n-1}\\
                \end{split}
            \end{equation*}
            Thus
            $$\lim (a_n)=0\neq2=\lim (b_n)$$
            Therefore, we know the sequence is divergent since Divergence criteria.
        }
        \subsection{}
        \paragraph{
            Let$(a_n)=(x_{4n})$ and $(b_n)=(x_{8n-1})$ be the subsequences of $(x_n)$.
        }
        \paragraph{
            Thus
            \begin{equation*}
                \begin{split}
                    a_n=&x_{4n}\\
                        =&\sin(\frac{4n\pi}{4})\\
                        =&\sin(n\pi)\\
                        =&0\\
                    b_n=&x_{8n-1}\\
                        =&\sin (\frac{(8n+1)\pi}{4})\\
                        =&\frac{\pi}{4}\\
                        =&\frac{\sqrt{2}}{2}
                \end{split}
            \end{equation*}
        }
        \paragraph{
            Therefore,
            $$\lim (a_n)=0\neq\frac{\sqrt{2}}{2}=\lim (b_n)$$
            which means that the sequence is divergent.
        }
    \section{}
        \paragraph{
            Because $x_n$ is unbouned sequence, then there exist a subsequence $(x_{n_k})$
            where
            $$|x_{n_k}| > k$$
            i.e.
            $$0\leq \frac{1}{|x_{n_k}|}\leq \frac{1}{k}$$
        }
        \paragraph{
            By the sandwich theorem
            $$0= \lim 0 \leq \lim \frac{1}{|x_{n_k}|} \leq \lim \frac{1}{k}=0 $$
            i.e.
            $$\lim \frac{1}{|x_{n_k}|}=0$$
        }
    
    \section{}
        \paragraph{
            By the definition of supermum,$\forall \epsilon>0,\exists x_n$ that 
            $$s-\epsilon<x_n<s$$
        }

        \paragraph{
            There must exist a subsequence $(x_{n_k})$
            that
            $$s-\frac{1}{k}<x_{n_k}<s$$
            Therefore,by the sandwich theorem,
            $$s=\lim_{k\rightarrow \infty} (s-\frac{1}{k})<\lim_{k\rightarrow \infty}x_{n_k}<\lim_{k\rightarrow \infty}s=s$$
            Thus,
            $$\lim(x_{n_k})=s$$
        }
    \section{}
        \paragraph{
            $$x^3-5x+1=0$$
            i.e.
            $$x=\frac{x^3+1}{5}$$
            Let $(x_n)$ be a sequence that:
            $$x_{n+1}=\frac{x_n^3+1}{5}$$
            where $0<x_1<5$
        }
        \paragraph{
            Because $0<x_1<1$, then $x_1^3+1<2$ and $x_{2}=\frac{x_1^3+1}{5}<\frac{2}{5}<1$\\
            And assume that $0<x_n<1$,then $x_n^3+1<2$ and $x_{n+1}=\frac{x_n^3+1}{5}<\frac{2}{5}<1$\\
            Thus, $x_n<1,\forall n$ and it is easy to know that $0<x_n$.\\
            It is easy to know that 
            $$\frac{x_{n+1}^2+x_{n+1}x_n+x_n^2}{5}<\frac{3}{5}<1$$
        }
            \begin{equation*}
                \begin{split}
                    |x_{n+2}-x_{n+1}|=&|\frac{x_{n+1}^3+1}{5}-\frac{x_n^3+1}{5}|\\
                        =&|\frac{x_{n+1}^3-x_n^3}{5}|\\
                        =&|\frac{(x_{n+1}-x_n)(x_{n+1}^2+x_{n+1}x_n+x_n^2)}{5}| \\ 
                        <&\frac{3}{5}|x_{n+1}-x_n|
                \end{split}
            \end{equation*}
        \paragraph{
            which shows that $(x_n)$ is contractive and the constant of contractive sequence $C=\frac{3}{5}$.\\
            We know that 
        }
            \begin{equation*}
                \begin{split}
                    |r-x_n|\leq& \frac{C}{1-C}|x_n-x_{n-1}|\\
                        =&\frac{3}{2}|x_n-x_{n-1}|\\
                \end{split}
            \end{equation*}
        \paragraph{
            Let $x_1=0.5$
        }
            \begin{equation*}
                \begin{split}
                    x_2=& \frac{9}{40}=0.2250000\\
                    |r-x_2|\leq& \frac{3}{2}|x_2-x_1|=\frac{33}{80}=0.4125000\\
                    x_3\approx& 0.2022781\\
                    |r-x_3|\leq& \frac{3}{2}|x_3-x_2|\approx0.0340828\\
                    x_4\approx& 0.2016553\\
                    |r-x_4|\leq& \frac{3}{2}|x_4-x_3|\approx0.0009342\\
                    x_5\approx& 0.2016401\\
                    |r-x_5|\leq& \frac{3}{2}|x_5-x_4|\approx0.0000229\\
                \end{split}
            \end{equation*}
        \paragraph{
            $0.2016$ is the approximation of $r$ within $10^{-4}$
        }
    
    \section{}
    \begin{proof}
        
    \end{proof}
    \section{}
        \subsection*{(a)}
        \begin{proof}
            \begin{equation*}
                \begin{split}
                    \lim a_n=&\sum_{k=1}^\infty \frac{1}{k}\\
                        \geq& 1+(\frac{1}{2}+\frac{1}{2})+(\frac{1}{4}+\frac{1}{4}+\frac{1}{4}+\frac{1}{4})...\\
                        =&1+1+1+...+1\\
                        =\infty
                \end{split}
            \end{equation*}
            Thus $(a_n)$ is divergence.
        \end{proof}
        \subsection*{b}
        \begin{proof}
            It is easy to know that
            $$\ln(x+1)=\sum_{n=1}^\infty\frac{(-1)^{n+1}}{n}x^n$$
            by Taylor expansion. When $x=1$,
            $$\ln(2)=\sum_{n=1}^\infty \frac{(-1)^{n+1}}{n}$$
            Thus,
            $$\lim a_n=\sum_{n=1}^\infty \frac{(-1)^{n+1}}{n}=\ln(2)$$
            $a_n$ is convergence.
        \end{proof}
    \section{}
        \paragraph{
            $$x_n=\frac{1}{2}((1+(-1)^n)n+(1+(-1)^{n+1})\frac{1}{n})$$
        }
\end{document}
\documentclass{article}
\usepackage{amsmath}
\usepackage{amssymb}
\usepackage{amsthm}                                        
\usepackage{textcomp}     
\usepackage{xcolor}
\usepackage{listings}   
\usepackage[linesnumbered,ruled,vlined]{algorithm2e}                   
\title{Assignment 1 of CISC 3000}
\author{ZHANG HUAKANG}
\begin{document}
    \maketitle
    \section{}
    \begin{itemize}
        \item $T(n)=3n^2+5n\log_2n=O(n)$. False
        \item $T(n)=4^{\log_2n}+\sqrt{n}=\Omega(n^2)$. True
        \item $T(n)=3n^2+9n=O(n^3)$. True
        \item $T(n)=4(\log_2n)^5+5\sqrt{n}+10=\Theta(\sqrt{n})$. False
        \item $T(n)=(\log_2n)^{\log_2n}+n^4=\Theta(n^4)$. True
    \end{itemize}
    \section{}
    \begin{algorithm}[H]
        \caption{Sum of three(A, K)}
        let $n \leftarrow |A|$ and assume $A = \{a_1, a_2, . . . , a_n\}$.

        \For(){$i==1,2,...,n-2$}{
            $j\leftarrow i+1$ and $k=n$.

            \While(){$k>j$}{
                \uIf{$a_i+a_j+a_k=K$}{
                    Output: $(i,j,k)$
                }
                \uElseIf{$a_i+a_j+a_k<K$}{
                    $j\leftarrow j+1$
                }
                \uElseIf{$a_i+a_j+a_k>K$}{
                    $k\leftarrow k-1$
                }
            }
        }
        Output: do not exist
    \end{algorithm}
    \newpage
    \section{}
    \subsection{}
    \begin{lstlisting}
             ______2___
            /          \
        ____3___        _9
       /        \      /  \
     _15        _13   36   10
    /   \      /
   44    29   22
    \end{lstlisting}
    \subsection{}
    \begin{lstlisting}
         ____13___
        /         \
      _15         _29
     /   \       /
    44    22    36
    \end{lstlisting}
    \newpage
    \section{}
    \subsection{}
    \begin{lstlisting}
          ______22__
         /          \
    ____15__         36
   /        \       /  \
 _44         10    3    9
/   \       /
13    29    2

          ______22__
         /          \
    ____15___        36
   /         \      /  \
 _13         _2    3    9
/   \       /
44    29    10

          ______22___
         /           \
    ____2___         _3
   /        \       /  \
 _13        _10    36   9
/   \      /
44    29   15

          _______2___
         /           \
    ____10___        _3
   /         \      /  \
 _13         _15   36   9
/   \       /
44    29    22
    \end{lstlisting}
    \newpage
    \subsection{}
    \begin{lstlisting}           
          _______2___
         /           \
    ____10___        _3
   /         \      /  \
 _13         _15   36   9
/   \       /
44    29    22

          ____3___
         /        \
    ____13        _9
   /      \      /  \
 _15       22   36   10
/   \
44    29

       ____9___
      /        \
    _13        _10
   /   \      /   \
 _15    22   36    29
/
44

    ____10___
   /         \
 _13         _29
/   \       /   \
15    22    36    44
    \end{lstlisting}
    

\section{}
\subsection{}
\begin{lstlisting}
         __________62
        /            \
   ____50___          78
  /         \           \
 _44         _55          88
/   \       /   \
17    48    54    60
\end{lstlisting}
\subsection{}
\begin{lstlisting}
     ____50______
    /            \
  _44            _60
 /   \          /   \
17    48      _55    88
             /
            54
\end{lstlisting}
\section{}
\begin{lstlisting}[language=Python]
def findAll(k1:int,k2:int):
    return _findAll(k1,k2,tree.root)
def _findAll(k1:int,k2:int,node:Node):
    if node is None:
        return None
    if node.val<k1:
        if node.right is None:
            return None
        else:
            return _findAll(k1,k2,node.right)
    elif node.val>k2:
        if node.left in None:
            return None
        else:
            return _findAll(k1,k2,node.left)
    else:
      return {
          _findAll(k1,k2,node.left),
          node.key,
          _findAll(k1,k2,node.right)  
      }
\end{lstlisting}
Suppose we have a BST:
\begin{lstlisting}
         ____25______
        /            \
   ____20            _36___
  /      \          /      \
 10       22      _30      _40
/  \             /        /   \
5    12          28       38    48
\end{lstlisting}
Let $k_1=11$ and $k_2=37$
\begin{lstlisting}
    |     ____25______    |
    |    /            \   |
   _|___20            _36_|__
  / |     \          /    |  \
 10 |      22      _30    |  _40
/  \|             /       | /   \
5   |12          28       |38    48
    |                     |
    11                    37
\end{lstlisting}
We will check node $25$, $20$, $10$ and $12$ at first. And add $12$ to the result.
\begin{lstlisting}
    |     ____|25|____    |
    |    /            \   |
   _|___|20|          _36_|__
  / |     \          /    |  \
|10||      22      _30    |  _40
/  \|             /       | /   \
5   ||12|        28       |38    48
    |                     |
    11                    37
\end{lstlisting}
Check node $22$ and add $20$, $22$ and $25$ to the result.
\begin{lstlisting}
    |     ____|25|____    |
    |    /            \   |
   _|___|20|          _36_|__
  / |     \          /    |  \
 10 |      |22|    _30    |  _40
/  \|             /       | /   \
5   |12          28       |38    48
    |                     |
    11                    37
\end{lstlisting}
Check node $36$ $30$ and $28$ and add $28$ $30$ and $36$ to the result.
\begin{lstlisting}
    |     ____25______    |
    |    /            \   |
   _|___20            _|36|__
  / |     \          /    |  \
 10 |      22      _|30|  |  _40
/  \|             /       | /   \
5   |12          |28|     |38    48
    |                     |
    11                    37
\end{lstlisting}
\newpage
Check node $40$ and $38$ and end the process.
\begin{lstlisting}
    |     ____25______    |
    |    /            \   |
   _|___20            _36_|__
  / |     \          /    |  \
 10 |      22      _30    |  _|40|
/  \|             /       | /   \
5   |12          28       |38|    48
    |                     |
    11                    37
\end{lstlisting}
\begin{proof}
    The first thing we need to do is find the leftmost node and rightmost in BST that they are in the range $[k_1,k_2]$. The worst case is that all node in BST is in the range. So find each node we need $O(\log n)$ and two nodes is $2O(\log n)$.

    The second step us add all node between leftmost node and rightmost node into result. If we have $s$ node, we will need $O(s)$ to do this.

    Thus, the total complexity of this algorithm is $$2O(\log n)+s=O(\log n+s)$$

\end{proof}
\end{document}
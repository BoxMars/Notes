\documentclass{article}
\usepackage{amsmath}
\usepackage{amssymb}
\usepackage{amsthm}                                        
\usepackage{textcomp}  
\usepackage{xcolor,listings}
\lstset{upquote=true}                             
\title{Assignment 1 of CISC 3025}
\author{ZHANG HUAKANG}
\begin{document}
    \maketitle
    \section{}
    \subsection*{a}
    \begin{verbatim}
[a-z]*b
    \end{verbatim}
    \subsection*{b}
    \begin{verbatim}
\bgrottos\b.*\braven\b|\braven\b.*\bgrottos\b
    \end{verbatim}
    \subsection*{c}
    \begin{verbatim}
([a-zA-Z]+)\s+\1
    \end{verbatim}
    \section{}
    \subsection*{a}
    Do not contain any alphabet.
    \begin{verbatim}
12345, OK
123As, No
    \end{verbatim}
    \subsection*{b}
    Two digts/Two digts/Four digts.
    Date
    \begin{verbatim}
10/11/2021
    \end{verbatim}
    \subsection*{c}
    \paragraph{}
    \textbf{Case 1:}
    This string contains two substring, the first substring contains at least one $a$. The second substring is starting with a $b$ and following by at least one $a$. This string can contains at least one the second substring.
    \paragraph{}
    \textbf{Case 2:}
    This string contain nothing(\emph{null})
    \begin{verbatim}
aabaababaabaaa
    or
null
    \end{verbatim}
    \section{}
    \begin{verbatim}
e, 5, 6, 5, 4, 3, 4 
f, 4, 5, 4, 3, 4, 5
i, 3, 4, 3, 2, 3, 4
r, 2, 3, 2, 3, 4, 5
b, 1, 2, 3, 4, 5, 6   
0, 0, 1, 2, 3, 4, 5
   0, d, r, i, v, e  
    \end{verbatim}
    \begin{verbatim}
s, 6, 5, 4, 3, 2, 1
e, 5, 4, 3, 2, 1, 0
v, 4, 3, 2, 1, 0, 1
i, 3, 2, 1, 0, 1, 2
r, 2, 1, 0, 1, 2, 3
d, 1, 0, 1, 2, 3, 4
0, 0, 1, 2, 3, 4, 5
   0, d, r, i, v, e
    \end{verbatim}
    $drive$ is closer to $driver$ than $brife$. 
\section{}
    \begin{tabular}{ccccccc}
s& 6& $\downarrow$5& $\downarrow$4& $\downarrow$3& $\downarrow$2& \emph{\textbf{$\downarrow$1}}\\
e& 5& $\downarrow$4& $\downarrow$3& $\downarrow$2& $\downarrow$1& \emph{\textbf{$\swarrow$0}}\\
v& 4& $\downarrow$3& $\downarrow$2& $\downarrow$1& \emph{\textbf{$\swarrow$0}}&$\leftarrow$ 1\\
i& 3& $\downarrow$2& $\downarrow$1& \emph{\textbf{$\swarrow$0}}& $\leftarrow$1& $\leftarrow$2\\
r& 2& $\downarrow$1& \emph{\textbf{$\swarrow$0}}& $\leftarrow$1& $\leftarrow$2& $\leftarrow$3\\
d& 1& \emph{\textbf{$\swarrow$0}}& $\leftarrow$1& $\leftarrow$2& $\leftarrow$3& $\leftarrow$4\\
0& \emph{\textbf{0}}& 1& 2& 3& 4& 5\\
 & 0& d& r& i& v& e\\
    \end{tabular}
    \section{}
    \subsection*{a}
    \begin{equation*}
        \begin{split}
            |V|=&6\\
            P(do|<s>)=&\frac{2}{11}\\
            P(do|Same)=&\frac{1}{11}\\
            P(Sam|<s>)=&\frac{4}{11}\\
            P(Sam|do)=&\frac{1}{8}\\
            P(I|Sam)=&\frac{4}{11}\\
            P(I|do)=&\frac{2}{8}\\
                    =&\frac{1}{4}\\
            P(like|I)=&\frac{3}{11}\\
        \end{split}
    \end{equation*}
    \subsection*{b}
    \begin{equation*}
        \begin{split}
            P(x|w)=\frac{del[w_i-1,w_i]+1}{c(w_{i-1,w_i)+|V|}}\\
            P(do \ Sam \ I \ like)=&P(do|<s>)\times P(Sam|do)\\
                            &\times P(I|do\ Sam)\times P(like|do\ Sam\ I)\\
                            \approx&P(do|<s>)\times P(Sam|do)\\
                            &\times P(I|Sam)\times P(like|I)\\
                            =&\frac{2}{11}\times\frac{1}{8}\times\frac{4}{11}\times\frac{3}{11}\\
                            =&\frac{3}{1331}
        \end{split}
    \end{equation*}
    \begin{equation*}
        \begin{split}
            P(Sam \ do\  I \ like)=&P(Sam|<s>)\times P(do|Sam)\\
                                =&P(I|do)\times P(like|I)\\
                                =&\frac{4}{11}\times\frac{1}{11}\times\frac{2}{8}\times\frac{3}{11}\\
                                =&\frac{3}{1331}
        \end{split}
    \end{equation*}
    
\end{document}
\documentclass[titlepage]{article}
\title{Notes of\\ Formal Laguage and Automata\\ CISC 3007}
\author{Box, ZHANG Huakang}
\begin{document}
    \maketitle
    \section{Basic Definitions and Properties}
        \paragraph{Alphabets}
        \begin{itemize}
            \item An alphabet is a finite set of symbols. 
            \item Usually use $\Sigma$ to represent an alphabet.
        \end{itemize}
        \subsection*{Strings}
            \paragraph{Definition}
            \begin{itemize}
                \item A string is a finite sequence of symbols feom an alphabet.
            \end{itemize}
            \paragraph{String Operations}
            \begin{itemize}
                \item Length: $|1100|=4$
                \item Prefix
                \item Suffix
                \item Substring
                \item Concarenation: $\alpha=abd,\beta=ce,\alpha\beta=abdce$
                \item Exponentiation: $\alpha=abd, \alpha^3=abdabdabd, \alpha^0=\epsilon$
                \item Reversal: $\alpha=abd, \alpha^{Rev}=dba$
                \item Power of an alphabet: $\Sigma^k$ is the set of all $k$-length strings formed by the alphabet in $\Sigma$. e.g., $\Sigma=\{a,b\}$, $\Sigma^2=\{ab,aa,bb,ba\}$, $\Sigma^0=\{\epsilon\}$
            \end{itemize}

\end{document}   